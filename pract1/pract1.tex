%% LyX 2.3.3 created this file.  For more info, see http://www.lyx.org/.
%% Do not edit unless you really know what you are doing.
\documentclass[english]{article}
\usepackage[T1]{fontenc}
\usepackage[latin9]{inputenc}
\usepackage{graphicx}

\makeatletter
%%%%%%%%%%%%%%%%%%%%%%%%%%%%%% Textclass specific LaTeX commands.
\newenvironment{lyxcode}
	{\par\begin{list}{}{
		\setlength{\rightmargin}{\leftmargin}
		\setlength{\listparindent}{0pt}% needed for AMS classes
		\raggedright
		\setlength{\itemsep}{0pt}
		\setlength{\parsep}{0pt}
		\normalfont\ttfamily}%
	 \item[]}
	{\end{list}}

\makeatother

\usepackage{babel}
\usepackage{listings}
\renewcommand{\lstlistingname}{Listing}

\begin{document}
\section{\label{cha:Example-Circuits}Example Circuits}
Lets start with an ngspice example to walk you through the basic features
of ngspice using its command line user interface. The operation of
ngspice will be illustrated through the example.

The example uses the simple one-transistor amplifier circuit illustrated
in Fig. \ref{cap:Transistor-Amplifier-Simulation}. 
\begin{figure}[ht!]
\includegraphics[width=0.8\linewidth]{Images/Example_Circuit_C1}

\caption{\label{cap:Transistor-Amplifier-Simulation}Transistor Amplifier Simulation
Example}
\end{figure}
This circuit is constructed entirely with ngspice compatible devices
and is used to introduce basic concepts, including:
\begin{itemize}
\item Invoking the simulator:
\item Running simulations in different analysis modes
\item Printing and plotting analog results
\item Examining status, including execution time and memory usage
\item Exiting the simulator
\end{itemize}

\section{\label{sec:AC-coupled-transistor}AC coupled transistor amplifier}

The circuit shown in Fig. \ref{cap:Transistor-Amplifier-Simulation}
is a simple one-transistor amplifier. The input signal is amplified
with a gain of approximately -(Rc/Re) = -(3.9K/1K) = -3.9. The circuit
description file for this example is shown below.

\noindent\begin{minipage}[t]{1\columnwidth}%
Example:
\begin{lyxcode}
\begin{lstlisting}
A Berkeley SPICE3 compatible circuit  
*  
* This circuit contains only Berkeley SPICE3 components.  
*  
* The circuit is an AC coupled transistor amplifier with  
* a sinewave input at node "1", a gain of approximately -3.9,  
* and output on node "coll".  
*  
.tran 1e-5 2e-3  
*  
vcc vcc 0 12.0  
vin 1 0 0.0 ac 1.0 sin(0 1 1k)  
ccouple 1 base 10uF  
rbias1 vcc base 100k 
rbias2 base 0 24k  
q1 coll base emit generic
rcollector vcc coll 3.9k  
remitter emit 0 1k
*
.model generic npn  
*    
.end
\end{lstlisting}
\end{lyxcode}
%
\end{minipage}

To simulate this circuit, invoke the simulator on this circuit as
follows:
\begin{lyxcode}
\$~ngspice~xspice\_c1.cir
\end{lyxcode}
After a few moments, you should see the ngspice prompt:
\begin{lyxcode}
ngspice~1~->
\end{lyxcode}
At this point, ngspice has read-in the circuit description and checked
it for errors. If any errors had been encountered, messages describing
them would have been output to your terminal. Since no messages were
printed for this circuit, the syntax of the circuit description was
correct.

To see the circuit description read by the simulator you can issue
the following command:
\begin{lyxcode}
ngspice~1~->~listing~
\end{lyxcode}
The simulator shows you the circuit description currently in memory: 
\begin{lyxcode}
	a~berkeley~spice3~compatible~circuit~~

~1~:~a~berkeley~spice3~compatible~circuit~~~~~~~~

~2~:~.global~gnd~~~~~

10~:~.tran~1e-5~2e-3~~~~~

12~:~vcc~vcc~0~12.0~~~~~

13~:~vin~1~0~0.0~ac~1.0~sin(0~1~1k)~~~~~

14~:~ccouple~1~base~10uf~~~~~

15~:~rbias1~vcc~base~100k~~~~~

16~:~rbias2~base~0~24k~~~~~

17~:~q1~coll~base~emit~generic~~~~~

18~:~rcollector~vcc~coll~3.9k~~~~~

19~:~remitter~emit~0~1k~~~~~

21~:~.model~generic~npn~~~~~

24~:~.end
\end{lyxcode}
The title of this circuit is `A Berkeley SPICE3 compatible circuit'.
The circuit description contains a transient analysis control command
\texttt{.TRAN 1E-5 2E-3} requesting a total simulated time of 2ms
with a maximum time-step of 10us. The remainder of the lines in the
circuit description describe the circuit of Fig. \ref{cap:Transistor-Amplifier-Simulation}.

Now, execute the simulation by entering the \texttt{run} command:
\begin{lyxcode}
ngspice~1~->~run~
\end{lyxcode}
The simulator will run the simulation and when execution is completed,
will return with the ngspice prompt. When the prompt returns, issue
the rusage command again to see how much time and memory has been
used now.
\begin{lyxcode}
\end{lyxcode}
To examine the results of this transient analysis, we can use the
plot command. First we will plot the nodes labeled `\texttt{1}' and
`base'.
\begin{lyxcode}
ngspice~2~->~plot~v(1)~base~
\end{lyxcode}
The simulator responds by displaying an X Window System plot similar
to that shown in Fig. \ref{fig:node-1-and}.

\begin{figure}[h]
\includegraphics{Images/C4}

\caption{\label{fig:node-1-and}node 1 and node 'base' versus time}
\end{figure}

Notice that we have named one of the nodes in the \emph{circuit description}
with a number (`\texttt{1}'), while the others are words (`base').
This was done to illustrate ngspice's special requirements for plotting
nodes labeled with numbers. Numeric labels are allowed in ngspice
for backwards compatibility with SPICE2. However, they require special
treatment in some commands such as \texttt{plot}. The \texttt{plot}
command is designed to allow expressions in its argument list in addition
to names of results data to be plotted. For example, the expression
\texttt{plot (base - 1)} would plot the result of subtracting 1 from
the value of node `base'.

If we had desired to plot the difference between the voltage at node
`base' and node `\texttt{1}', we would need to enclose the node name
`\texttt{1}' in the construction \texttt{v( )} producing a command
such as \texttt{plot (base - v(1))}.

Now, issue the following command to examine the voltages on two of
the internal nodes of the transistor amplifier circuit:
\begin{lyxcode}
ngspice~3~->~plot~vcc~coll~emit~
\end{lyxcode}
The plot shown in Fig. \ref{fig:VCC,-Collector-and} should appear.
Notice in the circuit description that the power supply voltage source
and the node it is connected to both have the name `vcc'. The \texttt{plot}
command above has plotted the node voltage `vcc'. However, it is also
possible to plot branch currents through voltage sources in a circuit.
ngspice always adds the special suffix \texttt{\#branch} to voltage
source names. Hence, to plot the current into the voltage source named
\textsf{vcc}, we would use a command such as \texttt{plot vcc\#branch}.

\begin{figure}[h]
\includegraphics{Images/C5}

\caption{\label{fig:VCC,-Collector-and}VCC, Collector and Emitter Voltages}
\end{figure}

Now let's run a simple DC simulation of this circuit and examine the
bias voltages with the \texttt{print} command. One way to do this
is to quit the simulator using the \texttt{quit} command, edit the
input file to change the \texttt{.tran} line to \texttt{.op} (for
'operating point analysis'), re-invoke the simulator, and then issue
the \texttt{run} command. However, ngspice allows analysis mode changes
directly from the ngspice prompt. All that is required is to enter
the control line, e.g. op (without the leading `.'). ngspice will
interpret the information on the line and start the new analysis run
immediately, without the need to enter a new \texttt{run} command.

To run the DC simulation of the transistor amplifier, issue the following
command:
\begin{lyxcode}
ngspice~4~->~op~
\end{lyxcode}
After a moment the ngspice prompt returns. Now issue the \texttt{print}
command to examine the emitter, base, and collector DC bias voltages.
\begin{lyxcode}
ngspice~5~->~print~emit~base~coll~
\end{lyxcode}
ngspice responds with:
\begin{lyxcode}
emit~=~1.293993e+00~base~=~2.074610e+00~coll~=~7.003393e+00
\end{lyxcode}
To run an AC analysis, enter the following command:
\begin{lyxcode}
ngspice~6~->~ac~dec~10~0.01~100~
\end{lyxcode}
This command runs a small-signal swept AC analysis of the circuit
to compute the magnitude and phase responses. In this example, the
sweep is logarithmic with `decade' scaling, 10 points per decade,
and lower and upper frequencies of 0.01 Hz and 100 Hz. Since the command
sweeps through a range of frequencies, the results are vectors of
values and are examined with the \texttt{plot} command. Issue to the
following command to plot the response curve at node `coll':
\begin{lyxcode}
ngspice~7~->~plot~coll~
\end{lyxcode}
This plot shows the AC gain from input to the collector. (Note that
our input source in the circuit description `vin' contained parameters
of the form `AC 1.0' designating that a unit-amplitude AC signal was
applied at this point.) For plotting data from an AC analysis, ngspice
chooses automatically a logarithmic scaling for the frequency (x)
axis.

To produce a more traditional `Bode' gain phase plot (again with automatic
logarithmic scaling on the frequency axis), we use the expression
capability of the \texttt{plot} command and the built-in Nutmeg functions
db( ) and ph( ):
\begin{lyxcode}
ngspice~8~->~plot~db(coll)~ph(coll)
\end{lyxcode}
The last analysis supported by ngspice is a swept DC analysis. To
perform this analysis, issue the following command:
\begin{lyxcode}
ngspice~9~->~dc~vcc~0~15~0.1~
\end{lyxcode}
This command sweeps the supply voltage `vcc' from 0 to 15 volts in
0.1 volt increments. To plot the results, issue the command:
\begin{lyxcode}
ngspice~10~->~plot~emit~base~coll~
\end{lyxcode}
Finally, to exit the simulator, use the \texttt{quit} command, and
you will be returned to the operating system prompt.
\begin{lyxcode}
ngspice~11~->~quit~
\end{lyxcode}
So long.
\end{document}
